\documentclass{article}

\usepackage[english]{babel}
\usepackage[utf8]{inputenc}
\usepackage{amsmath,amssymb}
\usepackage{graphicx}
\usepackage{tikz}
\usetikzlibrary{shapes}
\usepackage{natbib}
\usepackage{hyperref}
\usepackage{listings}
\usepackage{blindtext}

\newcommand{\click}[2]{\href{http://#1}{\colorlet{temp}{.}\color{blue}{\underline{\color{temp}#2}}\color{temp}}}  
\newcommand{\clicks}[2]{\href{https://#1}{\colorlet{temp}{.}\color{blue}{\underline{\color{temp}#2}}\color{temp}}}  

\newcommand{\cmdname}[1]{\texttt{#1}}
\newcommand\footurl[1]{\footnote{\url{#1}}}
\newcommand\urllink[2]{#1\footurl{#2}}
\newcommand\foothref[3]{#1\footnote{\href{#2}{#3}}}
\usepackage{upquote}

\hypersetup{
    colorlinks=true,
    linkcolor=blue,
    filecolor=magenta,      
    urlcolor=cyan,
    pdftitle={Overleaf Example},
    pdfpagemode=FullScreen,
}
\usepackage{xcolor}
\usepackage{caption}


%\DeclareCaptionFormat{mycaption}{#1}

% Set the custom caption format for all listings
%\captionsetup[lstlisting]{format=mycaption}
\renewcommand{\lstlistingname}{}% Listing -> Algorithm
\renewcommand{\lstlistlistingname}{List of \lstlistingname s}% List of Listings -> List of Algorithms

\lstdefinestyle{csharpstyle}{
    language={[Sharp]C},
    backgroundcolor=\color{black}, % Background color of the code
    basicstyle=\color{white}\ttfamily,
    keywordstyle=\color{blue},
    commentstyle=\color{green},
    stringstyle=\color{orange},
    numberstyle=\tiny\color{white}, % Font color of line numbers
    numbers=left,
    breaklines=true,
    showstringspaces=false,
    tabsize=4,
    frame=leftline,
    framerule=2pt,
    captionpos=t, % Position of the caption (bottom)
    numbersep=10pt, % Space between line numbers and code
    framexleftmargin=10pt, % Width of the line number background
    rulecolor=\color{black}, % Color of the line separating code and line numbers
}
\lstset{
    numbers=none, % Disable line numbering globally
}
\title{Blog}

\author{B1ngster}
\date{}

\begin{document}

\maketitle
\newpage


\tableofcontents
\newpage

 



\section*{20 April 2024 - \\ Taming Stochastic Parrots}
\addcontentsline{toc}{section}{\protect\numberline{}20 April 2024 - Taming Stochastic Parrots}

When using ChatGPT, it is important to specify what date you want to use. Large language Models such as ChatGPT give probabilistic output. The output differs when given the same input. This is due to the way the data is encoded and the algorithm used to extract the data, such as cosine similarity over a high-dimensional space, which gives rise to hallucinations.  

A way of overcoming different outputs is possibly using conventional programming techniques of caching and hashing the output; this could offer a speed improvement and cost reduction over using ChatGPT. Using ChatGPT requires a user to pay; however, you can use open-source language models such as LLama2, but there will still be costs for hosting the models. Chatgpt gives free credit - but I think you need to pay, so I need to wait til I have more money to investigate this. There are open-source 



LLM are capable of reasoning; this can be demonstrated by asking a math question without asking it to reason. It will show that it's working out and usually provides better results—referred to as Chain of Thought Prompting. Better results can be achieved via few-shot learning; give the model an example of what you are looking for. This can be achieved on OpenAI's Playground and through the chat log, where the conversation between the user and assistant can be artificially created after ChatGPT. Several more sophisticated ways of achieving this include context to large language model through RAG. Using agent frameworks such as Langchain and Autogen. 



\section*{1 Feb 2024 - \\ Engineering with logic}
\addcontentsline{toc}{section}{\protect\numberline{1 Feb 2024 - \\ Engineering with logic}}

I have recently been doing a Functional Skills English class, yes my English is terrible, as it turned out, but I would like to point out that there was a fire alarm during the test, so I am doing a level one qualification. Fun is it gives me more time to enjoy learning about English. 

Interestingly, it's noted that English aligns with first-order logic, documents to be digested into knowledge graphs created via subject-predicate-object triples, which comply with the RDF standard. SPO triples would be easy to extract if natural language would be easy to extract, but we don't talk in simple noun-verb-adjectives, and the words often used are ambiguous and vague.


 When ontological modelling OntoClean provides metrics such as identity, unity, rigidity, and dependence, Welty and Andersen added two more meta-properties: permanence and actuality.
In semantic modelling, relational considerations such as symmetry, inversion, and transitivity influence the structure of the knowledge base 


Codd's First Normal Form closely resembles RDF triples
Codd's algebra are the selection, the projection, the Cartesian product (also called the cross product or cross join), the set union, and the set difference.
Appart 



\section*{31 Jan 2024 - \\ New Year the same me}
\addcontentsline{toc}{section}{\protect\numberline{}31 Jan 2024 - New Year the same me}

While studying for my Master's in Data Science, my tutor informed me that studying without the foundations in the School of Engineering would be difficult. Therefore, I have taken up studying Level 3 in Engineering. Previously, I studied Level 3 in computing, moving on to a BSc in web technologies; this gave me a good understanding of several data science techniques such as database programming and animation. However, this did not prepare me to study data science at the School of Engineering.

Engineering is from the Greek to create, and that is what we have been doing in this course using metalwork. I have been manually turning, milling and drilling to create a tool maker clamp. For the first assignment of this module, I got a distinction. Sadly, I only got a pass on the AutoCAD assignment because when I completed the assignment, I made additional layers to check the design and forgot to delete them. It was a pass, and now, we have passed the deadline. 

I took the Engineering Principles exam this month, and I fooked it, but it's best out of two - I need to revise the conversions of energy and Archimedes principle and provide answers in the correct units. A reoccurring exam question is about measuring the gradients of a line, which is used in linear functions and creates a straight line graph; having a good understanding of this was important too while doing my Master's and occurred during the Digital Image Processing exam, which the function was used to identify lines in images. Indeed, to solve many equations, transposing them into line equations allows a graph of a problem.

This week, I have been off college as it's a holiday before the new semester. I'm looking forward to new classes and sharing what I learn.


\newpage

 \end{document}
